\documentclass[11pt, a4paper]{article}

\usepackage[utf8]{inputenc}
\usepackage[T1]{fontenc}
\usepackage{amsmath, amssymb, amsthm, bm}
\usepackage{geometry}
\usepackage{booktabs}
\usepackage{hyperref}
\usepackage{xcolor}
\usepackage{enumitem}

\geometry{top=2.5cm, bottom=2.5cm, left=2.5cm, right=2.5cm}

\title{\textbf{Topological Vortex Reactor (TVR):}
\\[4pt]
\large{A Purely Quantum Model of Topological Parametric Catalysis}\\[2pt]
\normalsize{Nonlinear Quantum Oscillator with Helicity Coupling}}

\author{\textbf{Fedor Kapitanov} \\
\textit{Independent Researcher} \\
ORCID: 0009-0009-6438-8730 \\
\texttt{prtyboom@gmail.com}}

\date{December 2025}

\begin{document}

\maketitle

\begin{abstract}
We introduce a fully quantum mechanical model of the \textbf{Topological Vortex Reactor} (TVR) as a testbed for controlled enhancement of strongly suppressed quantum transitions, in particular tunneling between metastable states. The TVR core is represented by a nonlinear parametric oscillator coupled to a helicity operator of quantized electromagnetic modes on a tensor-product Hilbert space. Topological structure is encoded in helicity sectors, and the interaction makes the oscillator's effective potential depend on the helicity eigenvalue, so that topologically nontrivial EM configurations act as control parameters for the oscillator dynamics. Parametric resonance and bifurcation phenomena are analyzed within Floquet quantum mechanics. Within this framework, ``metric catalysis'' is defined as a modulation of quantum tunneling rates for an effective reaction coordinate in a time-periodically deformed potential barrier, without invoking any modification of spacetime geometry or fundamental constants. The construction is compatible with a causal--information foundation of quantum theory, in which $\hbar$ appears as minimal distinguishable action (PCOD), and yields a self-contained, falsifiable quantum dynamical model aimed at exploring how topology and parametric driving can be used to manipulate rare quantum events.
\end{abstract}

\tableofcontents

\section{Conceptual Foundation}

\subsection{Layered structure without classical fields}

We adopt a three-layer structure, formulated entirely within quantum mechanics:

\begin{enumerate}[label=\textbf{Layer \arabic* --}]
    \item \textbf{Foundational Ontology:} The quantum is an expression of minimal causal difference (PCOD) \cite{Kapitanov2025zenodo}. Planck's constant $\hbar$ is interpreted as the minimal action required to distinguish two alternatives.
    \item \textbf{Mathematical Model:} A nonlinear quantum parametric oscillator, coupled to a helicity operator of quantized electromagnetic modes, all defined on a tensor-product Hilbert space.
    \item \textbf{Quantum ``Catalysis'':} Tunneling of an effective reaction coordinate through a barrier periodically modulated by the oscillator's dynamics (Floquet tunneling). No classical fields or spacetime geometry are introduced.
\end{enumerate}

Throughout, we work in standard non-relativistic and second-quantized quantum mechanics, in flat spacetime. The term ``metric'' is used only as a convenient name for a particular quantum degree of freedom (an operator), not for a classical spacetime metric.

\section{Motivation and Potential Applications}
\label{sec:motivation}

The primary motivation of the TVR model is to provide a controlled quantum-theoretical setting where:
\begin{itemize}
    \item[(i)] a topological invariant (helicity) of electromagnetic modes,
    \item[(ii)] a nonlinear parametric oscillator, and
    \item[(iii)] a rare quantum event (tunneling through a barrier)
\end{itemize}
are coupled in a single, fully specified Hamiltonian system.

This is relevant for at least three broad reasons:

\begin{enumerate}[label=(\alph*)]
    \item \textbf{Control of rare quantum transitions.} Many processes of physical interest are dominated by exponentially small tunneling probabilities (between metastable states, across reaction barriers, etc.). Time-periodic driving is a known route to modify such probabilities (Floquet-assisted tunneling). The TVR framework provides a structured way to do this, where the driving itself is generated by a nonlinear oscillator controlled by a topological degree of freedom.

    \item \textbf{Topological control knobs in quantum devices.} Modern quantum platforms (superconducting circuits, trapped ions, optomechanics) already use parametric driving and mode coupling. A natural extension is to incorporate topological invariants (such as EM helicity) as additional control parameters. The present model clarifies, at the level of operators and states, how such a topological control would enter the effective dynamics.

    \item \textbf{Link to causal--information foundations.} Within the PCOD view \cite{Kapitanov2025zenodo}, $\hbar$ quantifies minimal distinguishable action. Periodic driving and Floquet structure naturally highlight how discrete quanta of action, $n\hbar\omega$, govern the energy exchange between modes. TVR thus serves as an explicit dynamical arena where the causal--information interpretation can be confronted with standard quantum predictions.
\end{enumerate}

We do \emph{not} assume any modification of fundamental constants or spacetime geometry. All nontrivial effects arise from structured Hamiltonians and time-periodic driving within conventional quantum mechanics.

\section{Nonlinear Quantum Parametric Oscillator}

\subsection{Definition of the oscillator mode}

The ``metric'' degree of freedom is represented by a single continuous quantum mode with canonical operators
\begin{equation}
\hat x, \hat p \quad \text{on } \mathcal{H}_X, 
\qquad [\hat x,\hat p] = i\hbar.
\end{equation}
We interpret $\hat x$ as a generalized coordinate (order parameter) encoding a collective configuration. Its free Hamiltonian is taken as a nonlinear oscillator:
\begin{equation}
\hat H_X^{(0)}
= \frac{\hat p^2}{2m}
+ \frac{1}{2} m\Omega_0^2 \hat x^2
+ \frac{\lambda_3}{4}\hat x^4
+ \frac{\lambda_5}{6}\hat x^6,
\label{eq:HX0}
\end{equation}
with parameters:
\begin{itemize}
    \item $m$: effective mass,
    \item $\Omega_0$: natural frequency,
    \item $\lambda_3,\lambda_5$: nonlinear coefficients (Duffing-type and higher-order terms).
\end{itemize}

In the absence of driving and coupling, $\hat H_X^{(0)}$ has a discrete spectrum $\{E_n^{(0)}\}$ with eigenstates $\{\ket{n}\}$:
\begin{equation}
\hat H_X^{(0)}\ket{n} = E_n^{(0)}\ket{n},\qquad n=0,1,2,\dots
\end{equation}

\subsection{Parametric modulation}

To model parametric excitation, we introduce a time-dependent quadratic term in $\hat x$:
\begin{equation}
\hat H_X(t)
= \frac{\hat p^2}{2m}
+ \frac{1}{2} m\Omega^2(t) \hat x^2
+ \frac{\lambda_3}{4}\hat x^4
+ \frac{\lambda_5}{6}\hat x^6,
\label{eq:HXt}
\end{equation}
with
\begin{equation}
\Omega^2(t) = \Omega_0^2\left[1 + h \cos(2\omega_p t)\right],
\label{eq:Omega_param}
\end{equation}
where:
\begin{itemize}
    \item $h$ is the dimensionless modulation amplitude,
    \item $\omega_p$ is the parametric pump frequency.
\end{itemize}

The total Hamiltonian for the $X$-mode is then periodic in time with period
\begin{equation}
T_p = \frac{\pi}{\omega_p}, \qquad \hat H_X(t+T_p)=\hat H_X(t).
\end{equation}

Near the primary parametric resonance,
\begin{equation}
\omega_p \approx \Omega_0,
\end{equation}
the dynamics of $\hat x$ acquire instabilities and bifurcations analogous to the classical Mathieu and Duffing equations \cite{LandauLifshitz1960, NayfehMook1979}, now in a quantum setting.

\subsection{Heisenberg-picture equation of motion}

In the Heisenberg picture, the operator $\hat x(t)$ satisfies
\begin{equation}
\frac{d^2 \hat x}{dt^2}
= -\frac{1}{m}\frac{\partial \hat H_X(t)}{\partial \hat x}.
\end{equation}
Explicitly:
\begin{equation}
m \ddot{\hat x}
+ m\Omega^2(t)\hat x
+ \lambda_3 \hat x^3
+ \lambda_5 \hat x^5
= 0.
\label{eq:x_Heisenberg}
\end{equation}
This is the operator analogue of a parametrically driven nonlinear oscillator. In many situations, one studies either:
\begin{itemize}
    \item the dynamics of expectation values $\bar x(t)=\langle \hat x(t)\rangle$ in a suitable class of states (coherent, squeezed, etc.),
    \item or the Floquet spectrum of the full time-periodic Hamiltonian $\hat H_X(t)$.
\end{itemize}

\subsection{Effective equation for expectation value}

Under a mean-field or semiclassical approximation in which higher-order cumulants are neglected, the expectation value $\bar x(t)=\langle \hat x (t)\rangle$ obeys approximately
\begin{equation}
\ddot{\bar x} + \Gamma \dot{\bar x} 
+ \Omega_0^2\left[1 + h\cos(2\omega_p t)\right]\bar x
+ \alpha_3 \bar x^3
+ \alpha_5 \bar x^5
\simeq 0,
\label{eq:x_mean}
\end{equation}
where $\Gamma$ is an effective damping (originating from coupling to other modes), and $\alpha_3,\alpha_5$ are renormalized nonlinearities. Introducing the dimensionless variables
\begin{equation}
\tau = \omega_p t,\qquad
X(\tau) = \frac{\bar x(t)}{x_0},
\end{equation}
we obtain
\begin{equation}
X'' + \Gamma' X' + \alpha X 
- \beta X^3 + \delta X^5
= h_{\mathrm{eff}} X \cos(2\tau),
\label{eq:parametric_osc}
\end{equation}
with primes denoting derivatives w.r.t. $\tau$, and dimensionless parameters $(\Gamma',\alpha,\beta,\delta, h_{\mathrm{eff}})$ expressed in terms of $(\Gamma,\Omega_0,\lambda_3,\lambda_5,x_0,h)$. This is the standard form of a parametrically driven nonlinear oscillator, suitable for Floquet and bifurcation analysis.

\section{Quantized Electromagnetic Modes and Helicity}

\subsection{Electromagnetic Hilbert space}

We quantize the electromagnetic field in a finite volume $V$ with periodic boundary conditions. The EM Hilbert space is
\begin{equation}
\mathcal{H}_{\text{EM}}=\bigotimes_{\mathbf{k},\lambda}
\mathcal{H}_{\mathbf{k},\lambda},
\end{equation}
where $\mathbf{k}$ labels modes and $\lambda\in\{+,-\}$ labels circular polarizations. The photon creation and annihilation operators satisfy
\begin{equation}
[\hat a_{\mathbf{k},\lambda},\hat a_{\mathbf{k}',\lambda'}^\dagger]
=\delta_{\mathbf{k}\mathbf{k}'}\delta_{\lambda\lambda'},\qquad
[\hat a_{\mathbf{k},\lambda},\hat a_{\mathbf{k}',\lambda'}]=0.
\end{equation}

The free EM Hamiltonian is
\begin{equation}
\hat H_{\text{EM}} =
\sum_{\mathbf{k},\lambda}
\hbar\omega_{\mathbf{k}}\,
\hat a_{\mathbf{k},\lambda}^\dagger\hat a_{\mathbf{k},\lambda},
\label{eq:HEM}
\end{equation}
where $\omega_{\mathbf{k}} = c|\mathbf{k}|$ for photons in vacuum.

\subsection{Helicity operator as a topological invariant}

We introduce an operator that measures the net helicity (or chirality) of the EM field:
\begin{equation}
\hat K 
= \sum_{\mathbf{k}} \kappa(\mathbf{k})
\left(
\hat a_{\mathbf{k},+}^\dagger\hat a_{\mathbf{k},+}
- \hat a_{\mathbf{k},-}^\dagger\hat a_{\mathbf{k},-}
\right),
\label{eq:K_def}
\end{equation}
where $\kappa(\mathbf{k})$ is a real weight factor (for instance, depending on direction or frequency). 

If the free EM Hamiltonian $\hat H_{\text{EM}}$ is rotationally invariant and does not mix polarizations, then
\begin{equation}
[\hat H_{\text{EM}},\hat K]=0,
\end{equation}
and $\hat K$ is a conserved quantity for the free field. We interpret eigenvalues of $\hat K$ as labels of \emph{topological sectors}:
\begin{equation}
\hat K \ket{\Psi_Q} = Q \ket{\Psi_Q},
\qquad Q\in\mathbb{Z}\ \text{or}\ \mathbb{R}.
\end{equation}

States with nonzero $Q$ model ``vortex-like'' or ``Hopfion-like'' EM configurations in a purely quantum sense (through helicity imbalance), without any reference to classical field lines.

\section{Coupling Between Oscillator and Helicity}

\subsection{Interaction Hamiltonian}

To couple the nonlinear oscillator to the EM helicity, we postulate an interaction Hamiltonian of the form
\begin{equation}
\hat H_{\text{int}} =
g_1 \hat x\otimes \hat K
+ g_2 \hat x^2\otimes \hat K
+ \dots,
\label{eq:Hint}
\end{equation}
where $g_1,g_2$ are coupling constants with dimensions of energy (or energy per length, etc.). The full Hamiltonian for the $X$ and EM subsystems is
\begin{equation}
\hat H_{X\text{--EM}}(t)
= \hat H_X(t)\otimes \mathbb{I}_{\text{EM}}
+ \mathbb{I}_X\otimes \hat H_{\text{EM}}
+ \hat H_{\text{int}}.
\label{eq:HXEM}
\end{equation}

If $[\hat H_{\text{int}},\hat K]=0$, then $\hat K$ remains a (partial) constant of motion even in the presence of interaction. In this case, the total Hilbert space decomposes into invariant subspaces labeled by the helicity eigenvalue $Q$:
\begin{equation}
\mathcal{H} =
\bigoplus_Q \mathcal{H}^{(Q)},
\end{equation}
and dynamics within each sector are governed by an effective Hamiltonian with a $Q$-dependent potential for $\hat x$.

\subsection{Effective Hamiltonian in a fixed helicity sector}

In a sector with fixed eigenvalue $\hat K\to Q$ (treating $Q$ as c-number), the interaction reduces to a time-independent correction:
\begin{equation}
\hat H_X^{(Q)}(t)
= \frac{\hat p^2}{2m}
+ \frac{1}{2} m \Omega^2(t) \hat x^2
+ \frac{\lambda_3}{4}\hat x^4
+ \frac{\lambda_5}{6}\hat x^6
+ g_1 Q \hat x + g_2 Q \hat x^2 + \dots
\label{eq:HXQ}
\end{equation}
Thus the helicity sector modifies both the equilibrium position and the effective frequency of the oscillator. The topological invariant $Q$ acts as a control parameter in the bifurcation diagram of the parametric oscillator.

\section{Quantum Tunneling Modulated by the Oscillator}

\subsection{Reaction coordinate and static barrier}

To model ``catalysis'' in a quantum-mechanical sense, we introduce an effective reaction coordinate $\hat q$ with conjugate momentum $\hat p_q$ and Hamiltonian
\begin{equation}
\hat H_Q^{(0)} = \frac{\hat p_q^2}{2M} + U(\hat q),
\label{eq:HQ0}
\end{equation}
where $U(q)$ contains a potential barrier separating two metastable regions. A concrete toy model convenient for numerics is a symmetric double-well:
\begin{equation}
U(q) = a\left(q^2 - q_0^2\right)^2,
\label{eq:double_well}
\end{equation}
with minima at $q=\pm q_0$ and a barrier of height $U(0)=a q_0^4$.
Tunneling between the wells corresponds to a quantum transition with amplitude determined by the WKB exponent in the absence of driving:
\begin{equation}
T_0(E) \sim \exp\left( -\frac{2}{\hbar}\int_{-q_b}^{q_b}\sqrt{2M[U(q)-E]}\,dq \right),
\label{eq:WKB_static}
\end{equation}
where $\pm q_b$ are the classical turning points for a given energy $E$ below the barrier.

\subsection{Coupling between oscillator and reaction coordinate}

We couple the oscillator to the reaction coordinate with an interaction
\begin{equation}
\hat H_{XQ} = \gamma\,\hat x \, f(\hat q),
\label{eq:HXQ_couple}
\end{equation}
where $\gamma$ is a coupling constant and $f(\hat q)$ is some operator-valued function (for example, $f(\hat q)=\hat q$ or a localized function near $q=0$). For the double-well model it is natural to take
\begin{equation}
f(\hat q) = \hat q,
\end{equation}
so that the coupling locally tilts the double-well potential depending on the sign and amplitude of $\hat x$.

The full Hamiltonian for the $X$ and $Q$ subsystems is then
\begin{equation}
\hat H_{XQ}(t)
= \hat H_X(t)\otimes\mathbb{I}_Q
+ \mathbb{I}_X\otimes \hat H_Q^{(0)}
+ \hat H_{XQ}.
\label{eq:HXQ_full}
\end{equation}

\subsection{Effective time-periodic barrier and Floquet tunneling}

If the oscillator is in a state where its expectation value $\bar x(t)=\langle\hat x(t)\rangle$ can be treated as a classical periodic function due to parametric resonance, we may approximate the interaction in the $Q$-sector by
\begin{equation}
\hat H_{Q}(t) \simeq
\frac{\hat p_q^2}{2M}
+ U(\hat q) + \gamma\,\bar x(t) f(\hat q).
\label{eq:HQt}
\end{equation}
Assume, for concreteness, that in a certain parameter regime the oscillator settles into a stable parametric orbit with dominant subharmonic response at frequency $\omega_p$:
\begin{equation}
\bar x(t) \approx X_0\cos(\omega_p t + \varphi_0),
\label{eq:x_cl}
\end{equation}
with amplitude $X_0$ determined by the nonlinear dynamics of \eqref{eq:parametric_osc}. Then the reaction Hamiltonian becomes
\begin{equation}
\hat H_{Q}(t) \simeq
\frac{\hat p_q^2}{2M}
+ U(\hat q) + \gamma X_0 \cos(\omega_p t + \varphi_0) f(\hat q),
\label{eq:HQt_explicit}
\end{equation}
which is periodic in time with period $T_p' = 2\pi/\omega_p$.

This is a standard setting for \emph{Floquet quantum mechanics} \cite{Floquet1883}:
\begin{equation}
\left[\hat H_{Q}(t) - i\hbar\frac{\partial}{\partial t}\right]\ket{\Phi_\alpha(t)} 
= \varepsilon_\alpha \ket{\Phi_\alpha(t)},
\label{eq:Floquet_eq}
\end{equation}
with Floquet states $\ket{\Phi_\alpha(t+T_p')}=\ket{\Phi_\alpha(t)}$ and quasienergies $\varepsilon_\alpha$ defined modulo $\hbar\omega_p$.

Quantum tunneling in this time-periodic potential corresponds to \emph{photon-assisted} or \emph{Floquet-assisted} tunneling. The transmission coefficient $T(E)$ becomes a sum over channels involving absorption or emission of $n$ quanta of the drive:
\begin{equation}
T(E) = \sum_{n=-\infty}^{+\infty} T_n(E), \qquad
T_n(E) \sim T_0(E+n\hbar\omega_p)\,\chi_n,
\label{eq:T_expansion}
\end{equation}
where $\chi_n$ encode overlap factors dependent on the drive amplitude $\gamma X_0$, the shape of $f(q)$, and the structure of the Floquet states. In this precise sense, the parametric oscillator modulates the barrier and can change tunneling rates. No fundamental constants or spacetime structures are altered.

\subsection{Dimensionless form for numerical simulations}

For numerical work it is convenient to introduce dimensionless variables. For the double-well \eqref{eq:double_well} with $f(q)=q$, define
\begin{equation}
q = q_0\,y,\quad
t = \frac{\tau}{\omega_p},\quad
\psi(q,t) = \frac{1}{\sqrt{q_0}}\phi(y,\tau),
\end{equation}
and parameters
\begin{equation}
\epsilon = \frac{\hbar\omega_p}{4 a q_0^4},\quad
\mu = \frac{\gamma X_0}{4 a q_0^3}.
\end{equation}
Then the time-dependent Schr\"odinger equation
\begin{equation}
i\hbar\frac{\partial}{\partial t}\psi(q,t) =
\left[
-\frac{\hbar^2}{2M}\frac{\partial^2}{\partial q^2}
+ a(q^2-q_0^2)^2 + \gamma X_0\cos(\omega_p t)\,q
\right]\psi(q,t)
\end{equation}
becomes, in dimensionless form,
\begin{equation}
i\epsilon\,\frac{\partial}{\partial \tau}\phi(y,\tau)
=
\left[
-\epsilon^2\frac{\partial^2}{\partial y^2}
+ (y^2-1)^2 + 4\mu \cos\tau\,y
\right]\phi(y,\tau).
\label{eq:dimless_SE}
\end{equation}
This is a convenient starting point for Floquet numerics: for given $(\epsilon,\mu)$ one can compute quasienergies and tunneling rates and study how they depend on the drive amplitude $\mu$.

\section{Floquet Bifurcations of the Oscillator Mode}

\subsection{Classical instability tongues and their quantum analogue}

For the classical equation \eqref{eq:parametric_osc}, the boundaries of the first instability tongue (near $\omega_p\approx\Omega_0$) are given approximately by \cite{LandauLifshitz1960, NayfehMook1979}
\begin{equation}
h_{\mathrm{crit}} \simeq \frac{2\Gamma'}{\sqrt{\alpha}},
\label{eq:hcrit}
\end{equation}
where $\Gamma'$ is the dimensionless damping and $\alpha$ is the effective linear stiffness. Quantum mechanically, these instabilities manifest as:
\begin{itemize}
    \item growth of occupation number of the oscillator,
    \item formation of squeezed states,
    \item rearrangement of the Floquet quasienergy spectrum of $\hat H_X(t)$,
    \item emergence of subharmonic response in $\langle\hat x(t)\rangle$ at frequency $\omega_p$.
\end{itemize}

\subsection{Subharmonic response}

The parametric drive at frequency $2\omega_p$ generates subharmonic responses at $\omega_p$ in the oscillator dynamics. This appears in:
\begin{itemize}
    \item spectral peaks in $\langle \hat x(t)\rangle$ at $\omega_p$,
    \item nontrivial structure in the Floquet eigenstates with period $2T_p$.
\end{itemize}
These features are purely quantum manifestations of the classical parametric resonance and can be analyzed via Floquet theory for $\hat H_X(t)$.

\section{Testable Predictions and Falsifiability}

Although the present work is primarily theoretical, it leads to several qualitative and quantitative predictions that, in principle, can be tested on suitable quantum platforms (e.g.\ superconducting circuits, trapped ions, optomechanics):

\begin{enumerate}[label=(P\arabic*)]
    \item \textbf{Subharmonic resonance of the nonlinear mode.}\\
    In regimes where the effective drive amplitude $h_{\mathrm{eff}}$ exceeds the critical value $h_{\mathrm{crit}}$ from \eqref{eq:hcrit}, the oscillator expectation value $\langle\hat x(t)\rangle$ exhibits a robust spectral peak at $\omega_p$ under pumping at $2\omega_p$. This is a clear signature of parametric resonance and bifurcation.

    \item \textbf{Helicity-dependent deformation of the oscillator potential.}\\
    In different helicity sectors $Q$ the effective Hamiltonian \eqref{eq:HXQ} predicts shifts of the oscillator frequency and equilibrium position. Preparing EM states with different net helicity should, in principle, shift the parametric resonance condition and modify the bifurcation diagram. Observation of such shifts would confirm the helicity coupling mechanism.

    \item \textbf{Enhancement or suppression of tunneling via parametric modulation.}\\
    For the double-well model \eqref{eq:dimless_SE}, increasing the drive amplitude $\mu$ at fixed $\epsilon$ should lead to a nontrivial dependence of the effective tunneling splitting between symmetric and antisymmetric states in the double well. In particular, one expects regimes of enhanced tunneling (reduced effective barrier) and regimes of coherent destruction of tunneling (CDT) for specific combinations of $(\epsilon,\mu)$.

    \item \textbf{Topological control of tunneling.}\\
    Since the oscillator amplitude $X_0$ and phase are influenced by the helicity sector $Q$ via \eqref{eq:HXQ}, the effective modulation parameter $\mu\propto\gamma X_0$ in \eqref{eq:dimless_SE} becomes a function of $Q$. Thus, in principle, preparing different EM helicity sectors should modulate the tunneling rate for the reaction coordinate, providing a direct test of ``topological parametric catalysis''.
\end{enumerate}

Failure to observe any dependence of the tunneling characteristics on parametric drive or helicity, in a regime where the model parameters are under control, would falsify the specific coupling scheme proposed here.

\section{Summary and Outlook}

We have formulated the TVR framework as a \emph{purely quantum} model, with the following key elements:
\begin{enumerate}
    \item A nonlinear quantum parametric oscillator $\hat x$ with time-periodic frequency $\Omega(t)$ and higher-order nonlinearity.
    \item A helicity operator $\hat K$ in the EM Hilbert space, defining topological sectors via its eigenvalues $Q$.
    \item An interaction $\hat H_{\text{int}}=\sum_n g_n \hat x^n\otimes\hat K$ that couples the oscillator to EM helicity, making topological charge a control parameter for the oscillator's effective potential.
    \item An effective reaction coordinate $\hat q$ experiencing a time-periodically modulated barrier $U(q)+\gamma\,\bar x(t) f(q)$, leading to Floquet-assisted tunneling.
\end{enumerate}

This construction realizes ``metric catalysis'' as a quantum phenomenon of tunneling in a time-periodically deformed potential landscape, without invoking any modification of spacetime geometry or fundamental constants. The PCOD foundation appears naturally via $\hbar$ in the Floquet quasienergy structure and in the discretized energy exchange $n\hbar\omega_p$.

\bigskip
\noindent
\textbf{Future work} could include:
\begin{itemize}
    \item Explicit numerical computation of Floquet spectra for $\hat H_X(t)$ and \eqref{eq:dimless_SE}, including tunneling splittings as functions of $(\epsilon,\mu,Q)$.
    \item Analysis of entanglement between $\hat x$, $\hat K$ and $\hat q$ during parametric driving.
    \item Investigation of optimal parameter regimes for enhanced or suppressed tunneling in this quantum parametric setting, and mapping to concrete experimental platforms.
\end{itemize}

\section*{Acknowledgments}
The author acknowledges discussions with independent research communities and the conceptual influence of the PCOD framework.

\begin{thebibliography}{99}

\bibitem{Kapitanov2025zenodo}
F.~Kapitanov, \emph{The Quantum as Minimal Difference: A Causal--Information Foundation}, Zenodo (2025), \url{https://zenodo.org/record/17794107}.

\bibitem{LandauLifshitz1960}
L.~D.~Landau and E.~M.~Lifshitz, \emph{Mechanics} (Pergamon, 1960).

\bibitem{NayfehMook1979}
A.~H.~Nayfeh and D.~T.~Mook, \emph{Nonlinear Oscillations} (Wiley, 1979).

\bibitem{Floquet1883}
G.~Floquet, \emph{Sur les équations différentielles linéaires à coefficients périodiques}, Ann. Éc. Norm. Supér. \textbf{12}, 47 (1883).

\end{thebibliography}

\end{document}
